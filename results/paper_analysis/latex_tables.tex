# LaTeX Tables for Publication - Chest X-Ray Classification

## Table 1: Performance Comparison of Individual Models and Ensemble Methods

```latex
\begin{table}[h]
\centering
\caption{Performance comparison of Transfer Learning models and Ensemble methods on the Chest X-Ray Pneumonia test set (N=624).}
\label{tab:performance_comparison}
\begin{tabular}{lccccc}
\hline
\textbf{Model} & \textbf{Accuracy} & \textbf{AUC} & \textbf{F1-Score} & \textbf{Sensitivity} & \textbf{Specificity} \\
\hline
\textbf{EfficientNet-B0} & \textbf{80.29\%} & \textbf{0.9761} & \textbf{0.8635} & 99.74\% & \textbf{47.86\%} \\
DenseNet-121 & 68.91\% & 0.9505 & 0.8008 & 100.00\% & 17.09\% \\
ResNet-50 & 67.15\% & 0.9230 & 0.7915 & 99.74\% & 12.82\% \\
Simple Voting & 71.47\% & 0.9742 & 0.8142 & 100.00\% & 23.93\% \\
Weighted Voting & 71.47\% & 0.9742 & 0.8142 & 100.00\% & 23.93\% \\
\hline
\end{tabular}
\end{table}
```

## Table 2: Confusion Matrix for EfficientNet-B0

```latex
\begin{table}[h]
\centering
\caption{Confusion matrix for EfficientNet-B0 on the test set (N=624).}
\label{tab:confusion_matrix}
\begin{tabular}{lcc}
\hline
 & \textbf{Predicted Normal} & \textbf{Predicted Pneumonia} \\
\hline
\textbf{Actual Normal (N=234)} & 112 (TN) & 122 (FP) \\
\textbf{Actual Pneumonia (N=390)} & 1 (FN) & 389 (TP) \\
\hline
\end{tabular}
\smallskip

\noindent\textbf{Derived Metrics:}
\begin{itemize}
    \item Positive Predictive Value (PPV): 76.13\%
    \item Negative Predictive Value (NPV): 99.12\%
    \item False Positive Rate: 52.14\%
    \item False Negative Rate: 0.26\%
\end{itemize}
\end{table}
```

## Table 3: Model Architecture Specifications

```latex
\begin{table}[h]
\centering
\caption{Architecture specifications of the three Transfer Learning models used in this study.}
\label{tab:model_specs}
\begin{tabular}{lccc}
\hline
\textbf{Model} & \textbf{Parameters} & \textbf{FLOPs} & \textbf{Key Feature} \\
\hline
EfficientNet-B0 & 5.3M & 0.39B & Compound scaling \\
ResNet-50 & 25.6M & 4.1B & Residual connections \\
DenseNet-121 & 8.0M & 2.9B & Dense connectivity \\
\hline
\end{tabular}
\end{table}
```

## Table 4: Dataset Distribution

```latex
\begin{table}[h]
\centering
\caption{Distribution of the Chest X-Ray Images (Pneumonia) dataset.}
\label{tab:dataset_distribution}
\begin{tabular}{lcccc}
\hline
\textbf{Split} & \textbf{Normal} & \textbf{Pneumonia} & \textbf{Total} & \textbf{Ratio} \\
\hline
Training & 1,341 & 3,875 & 5,216 & 1:2.89 \\
Validation & 8 & 8 & 16 & 1:1.00 \\
Test & 234 & 390 & 624 & 1:1.67 \\
\hline
\textbf{Total} & 1,583 & 4,273 & 5,856 & 1:2.70 \\
\hline
\end{tabular}
\end{table}
```

## Table 5: Training Configuration

```latex
\begin{table}[h]
\centering
\caption{Hyperparameters and training configuration for all models.}
\label{tab:training_config}
\begin{tabular}{ll}
\hline
\textbf{Parameter} & \textbf{Value} \\
\hline
Optimizer & AdamW \\
Learning Rate (initial) & $1 \times 10^{-4}$ \\
Weight Decay & $1 \times 10^{-4}$ \\
Batch Size & 32 \\
Image Size & $224 \times 224$ \\
Epochs (Phase 1) & 5 (classifier-only) \\
Epochs (Phase 2) & 20 (full fine-tuning) \\
LR Scheduler & ReduceLROnPlateau \\
Early Stopping Patience & 7 epochs \\
Class Weights & [1.945, 0.673] \\
Device & Apple Silicon (MPS) \\
\hline
\end{tabular}
\end{table}
```

## Table 6: Statistical Significance Testing

```latex
\begin{table}[h]
\centering
\caption{Statistical significance testing between EfficientNet-B0 and ensemble methods.}
\label{tab:statistical_tests}
\begin{tabular}{lcccc}
\hline
\textbf{Comparison} & \textbf{Test} & \textbf{Statistic} & \textbf{p-value} & \textbf{Significant?} \\
\hline
EfficientNet-B0 vs Simple Voting & McNemar's & $\chi^2 = 23.47$ & $1.28 \times 10^{-6}$ & Yes ($p < 0.001$) \\
EfficientNet-B0 vs Weighted Voting & McNemar's & $\chi^2 = 23.47$ & $1.28 \times 10^{-6}$ & Yes ($p < 0.001$) \\
Simple vs Weighted Voting & McNemar's & $\chi^2 = 0.00$ & 1.000 & No (identical) \\
\hline
\end{tabular}
\end{table}
```

## Table 7: Clinical Performance Thresholds

```latex
\begin{table}[h]
\centering
\caption{Clinical performance thresholds and model compliance.}
\label{tab:clinical_thresholds}
\begin{tabular}{lcccc}
\hline
\textbf{Metric} & \textbf{Threshold} & \textbf{EfficientNet-B0} & \textbf{Ensemble} & \textbf{Status} \\
\hline
Sensitivity & $\geq 95\%$ & 99.74\% & 100.00\% & ✓ Pass \\
Specificity & $\geq 60\%$ & 47.86\% & 23.93\% & ✗ Fail \\
AUC & $\geq 0.90$ & 0.9761 & 0.9742 & ✓ Pass \\
Balanced Accuracy & $\geq 75\%$ & 73.80\% & 61.97\% & ✗ Fail \\
\hline
\end{tabular}
\end{table}
```

## Table 8: Comparison with Literature

```latex
\begin{table}[h]
\centering
\caption{Comparison of our results with related work on chest X-ray pneumonia classification.}
\label{tab:literature_comparison}
\begin{tabular}{llccc}
\hline
\textbf{Study} & \textbf{Model} & \textbf{Accuracy} & \textbf{Sensitivity} & \textbf{Specificity} \\
\hline
\textbf{Our work} & EfficientNet-B0 & 80.29\% & 99.74\% & 47.86\% \\
Kermany et al. (2018) & Inception-v3 & 92.80\% & 93.20\% & 90.10\% \\
Rajpurkar et al. (2017) & CheXNet (DenseNet-121) & 76.80\% & N/A & N/A \\
Wang et al. (2017) & ChestX-ray14 & 73.40\% & N/A & N/A \\
\hline
\end{tabular}
\footnotesize
\textit{Note: Direct comparison is limited due to different datasets and evaluation protocols.}
\end{table}
```

## Table 9: Ensemble Weight Distribution

```latex
\begin{table}[h]
\centering
\caption{Weight distribution for Weighted Voting ensemble based on individual model AUC.}
\label{tab:ensemble_weights}
\begin{tabular}{lccc}
\hline
\textbf{Model} & \textbf{AUC} & \textbf{Weight} & \textbf{Percentage} \\
\hline
EfficientNet-B0 & 0.9761 & 0.3426 & 34.26\% \\
DenseNet-121 & 0.9505 & 0.3336 & 33.36\% \\
ResNet-50 & 0.9230 & 0.3238 & 32.38\% \\
\hline
\textbf{Total} & 2.8496 & 1.0000 & 100.00\% \\
\hline
\end{tabular}
\smallskip

\noindent\textit{Weight calculation: } $w_i = \frac{\text{AUC}_i}{\sum_j \text{AUC}_j}$
\end{table}
```

## Table 10: Training Time Analysis

```latex
\begin{table}[h]
\centering
\caption{Training time for each model on Apple Silicon (MPS backend).}
\label{tab:training_time}
\begin{tabular}{lccc}
\hline
\textbf{Model} & \textbf{Phase 1 (5 epochs)} & \textbf{Phase 2 (20 epochs)} & \textbf{Total} \\
\hline
EfficientNet-B0 & $\sim$25 min & $\sim$2h 0min & $\sim$2h 25min \\
ResNet-50 & $\sim$35 min & $\sim$2h 50min & $\sim$3h 25min \\
DenseNet-121 & $\sim$30 min & $\sim$2h 20min & $\sim$2h 50min \\
\hline
\textbf{Total} & - & - & $\sim$8h 40min \\
\hline
\end{tabular}
\end{table}
```

---

## Usage Instructions

1. Copy the desired LaTeX code into your document
2. Ensure you have the required packages:
   ```latex
   \usepackage{booktabs}  % For professional tables
   \usepackage{multirow}  % For multi-row cells
   \usepackage{amsmath}   % For mathematical symbols
   ```

3. Compile with pdflatex or xelatex

## Notes

- All tables use the `\begin{table}[h]` float specifier (here), adjust as needed
- Bold formatting indicates best performance in each category
- Statistical symbols (✓/✗) may need to be replaced with LaTeX equivalents
- Tables are formatted for two-column IEEE/ACM style papers
- For single-column format, you may need to adjust widths

## Citation Format

When referencing these tables in your paper:

```latex
As shown in Table~\ref{tab:performance_comparison}, EfficientNet-B0 
achieved the highest accuracy of 80.29\% among all models tested.

The confusion matrix (Table~\ref{tab:confusion_matrix}) reveals that 
the model correctly identified 389 out of 390 pneumonia cases.
```
